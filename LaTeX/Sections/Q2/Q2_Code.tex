\begin{lstlisting}[style=CStyle]
/**
 * Homework 3.2
 * EECE 5643 - Simulation and Performance Evaluation
 * Author: Harrison Sun
 * Email: sun.har@northeastern.edu
 */

#include <cstdlib>
#include <cstring>
#include <stdio.h>
#include <exception>
#include <iostream>
#include <math.h> 
#include <string>
#include "c_lib/rvgs.h"
#include "c_lib/rngs.h"

#define LAST         10000L                   /* number of jobs processed */
#define START        0.0                      /* initial time             */

 /**
  * double GetArrival()
  *
  * @param void
  * @return arrival - the next arrival time
  *
  * This function calculates the arrival times for each process.
  */

double GetArrival()
{
    static double arrival = START;

    arrival += Exponential(2.0);
    return (arrival);
}


/**
 * double GetService()
 *
 * @param void
 * @return sum - the total service time for the process
 *
 * This function calculates the service times for each process.
 */

double GetService()
{
    long k{};
    double sum{ 0.0 };
    long tasks = 1 + Geometric(0.9);
    for (k = 0; k < tasks; ++k)
    {
        sum += Uniform(0.1, 0.2);
    }
    return sum;
}

/**
 * bool checkArg()
 *
 * @param char* input - the input string literal from the console
 * @return bool - true if the input is a number, false otherwise
 *
 * This function determines whether the argument is a number.
 */

bool checkArg(char* input)
{
    try
    {
        if (strlen(input) > 9)
        {
            throw std::logic_error("Number is too large.");
        }

        for (int i = 0; i < strlen(input); ++i)
        {

            if (std::isdigit(input[i])) continue;
            else
            {
                std::string errorMessage;
                errorMessage.append((std::string)input);
                errorMessage.append(" is not a digit.");
                throw std::logic_error(errorMessage);
            }
        }
        return 1;
    }

    catch (const std::logic_error& error)
    {
        std::cerr << error.what() << std::endl;
        return 0;
    }
}

/**
 * int main()
 *
 * @param int argc - the number of arguments
 * @param char* argv[] - the arguments
 *
 * @return int - 0 if the program runs successfully
 */

int main(int argc, char* argv[])
{
    long   index = 0;                         /* job index            */
    double arrival = START;                     /* time of arrival      */
    double delay;                                 /* delay in queue       */
    double service;                               /* service time         */
    double wait;                                  /* delay + service      */
    double departure = START;                     /* time of departure    */
    struct {                                      /* sum of ...           */
        double delay;                               /*   delay times        */
        double wait;                                /*   wait times         */
        double service;                             /*   service times      */
        double interarrival;                        /*   interarrival times */
    } sum = { 0.0, 0.0, 0.0 };

    long numRuns{};                                  /* number of runs */

    // Set the seed
    for (int i = 0; i < argc; ++i)
    {
        if (*argv[i] == 's' && checkArg(argv[i + 1]))
        {
            PutSeed(std::stol(argv[i + 1]));
            break;
        }
        else
        {
            PutSeed(123456789);
        }
    }

    // Set the number of runs
    for (int i = 0; i < argc; ++i)
    {
        if (*argv[i] == 'r' && checkArg(argv[i + 1]))
        {
            numRuns = std::stol(argv[i + 1]);
            break;
        }
        else
        {
            numRuns = 10000;
        }
    }

    while (index < numRuns) {
        index++;
        arrival = GetArrival();
        if (arrival < departure)
            delay = departure - arrival;         /* delay in queue    */
        else
            delay = 0.0;                         /* no delay          */
        service = GetService();
        wait = delay + service;
        departure = arrival + wait;              /* time of departure */
        sum.delay += delay;
        sum.wait += wait;
        sum.service += service;
    }
    sum.interarrival = arrival - START;

    printf("\nfor %ld jobs\n", index);
    printf("   average interarrival time = %6.2f\n", sum.interarrival / index);
    printf("   average wait ............ = %6.2f\n", sum.wait / index);
    printf("   average delay ........... = %6.2f\n", sum.delay / index);
    printf("   average service time .... = %6.2f\n", sum.service / index);
    printf("   average # in the node ... = %6.2f\n", sum.wait / departure);
    printf("   average # in the queue .. = %6.2f\n", sum.delay / departure);
    printf("   utilization ............. = %6.2f\n", sum.service / departure);
    return 0;
}
\end{lstlisting}