\textbf{(a) Modify program ssq2 to use $Exponential(1.5)$ service times.}\\
\textbf{(b) Process a relatively large number of jobs, say 100000, and report what changes this produces relative to the statistics in Example 3.1.3.}\\
\textbf{(c) Explain (or conjecture) why some statistics change and others do not.}\\

\begin{table}[h]
\centering
\begin{tabular}{l|lllll}
                  & 123456 & 123456789 & 975312468 & 97531 & 246810 \\
                  \hline\\
interarrival time & 2.00   & 2.00      & 2.00      & 2.00  & 2.00   \\
wait time         & 5.99   & 6.04      & 6.10      & 5.95  & 6.01   \\
delay time        & 4.49   & 4.53      & 4.60      & 4.45  & 4.51   \\
service time      & 1.50   & 1.50      & 1.50      & 1.50  & 1.51   \\
number in node    & 3.00   & 3.02      & 3.05      & 2.97  & 3.00   \\
number in queue   & 2.25   & 2.27      & 2.30      & 2.22  & 2.25   \\
utilization       & 0.75   & 0.75      & 0.75      & 0.75  & 0.75  
\end{tabular}
\end{table}\\
\begin{center}
    \includegraphics[scale=0.75]{Sections/Q1/3.1.3.png}
\end{center}\\
\noindent The average interarrival time ($\Bar{r}$), average service time ($\Bar{s}$), and utilization ($\Bar{x}$) remain the same. The average wait time ($\Bar{w}$), average delay time ($\Bar{d}$), average number in the node ($\Bar{l}$), and average number in the queue ($\Bar{q}$) increase. This is because the service time distribution changes from a $Uniform(1.0,2.0)$ to an $Exponential(1.5)$. While the mean stays the same, the average service time increases due to the significantly higher possible service times caused by the exponential distribution. Whenever this happens, the number in the queue (and thus, the number in the node) increases and propagates to higher wait times as the subsequent jobs arrive.
\begin{center}
    \includegraphics[scale=1]{Sections/Q1/H3_1.png}
\end{center}
\include{Sections/Q1/Q1_Code.tex}