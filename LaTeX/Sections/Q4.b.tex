\textbf{(b) How does this probability depend on $\rho$?\\\\}
It is not affected by $\rho$.\\\\
We can see from the Monte Carlo Simulation that the probability of the distance being greater than the radius is approximately $\frac{2}{3}$. Furthermore, this probability is unaffected by the radius at all and the values are exactly the same given the same seed. This is because the chord length, which we can denote as $\mathcal{L}$ is proportional to the radius $\rho$ such that $\mathcal{L} = 2\times\rho\times sin(\frac{\theta}{2})$ where $\theta$ is the minor angle of the chord. We can represent this as any given point fixed in space on the circumference of the circle and the other point rotated about the center of the circle given by $\theta$ in either direction. Given this, we can say that we can consider a single direction and limit the angle to $\theta \leq \pi$. $\mathcal{L} > \rho $ when $sin(\frac{\theta}{2}) \geq \frac{1}{2}$. Thus, this occurs so long as $\theta \geq \frac{\pi}{3}$. Remembering that $\theta \sim  Uniform[0, \pi]$, this means that $\frac{2}{3}$ of the time, the length of the chord will be greater than the radius irrelevant of what the radius is.\\\\