\textbf{(a) Verify that the mean service time in Example 3.1.4 is 1.5.}\\
\textbf{(b) Verify that the steady-state statistics in Example 3.1.4 seem to be correct.}\\
\textbf{(c) Note that the arrival rate, service rate, and utilization are the same as those in Example 3.1.3. Explain (or conjecture) why this is so. Be Specific.}\\
\begin{center}
\includegraphics[scale=0.75]{Sections/Q2/3.1.3.png}\\
\vspace{10pt}
\includegraphics[scale=0.75]{Sections/Q2/3.1.4.png}\\
\end{center}
\newpage
\vspace{35pt}
\begin{center}
    \includegraphics[scale=0.75]{Sections/Q2/H3_2.png}
\end{center}
\newpage
\begin{table}[h]
\centering
\begin{tabular}{l|lllll}
                  & 123456 & 246810 & 1357911 & 987654321 & 12357 \\
                  \hline\\
interarrival time & 2.00   & 2.00   & 2.00    & 2.00      & 2.00  \\
wait time         & 5.61   & 5.82   & 5.58    & 5.73      & 5.66  \\
delay time        & 4.10   & 4.31   & 4.09    & 4.23      & 4.16  \\
service time      & 1.50   & 1.50   & 1.50    & 1.50      & 1.49  \\
number in node    & 2.80   & 2.90   & 2.79    & 2.86      & 2.82  \\
number in queue   & 2.05   & 2.15   & 2.04    & 2.11      & 2.08  \\
utilization       & 0.75   & 0.75   & 0.75    & 0.75      & 0.75 
\end{tabular}
\end{table}
\noindent The arrival rate is the same because the interarrival time is defined by the same distribution as in Example 3.1.3: $r \sim Exponential(2.0)$.\\\\

\noindent The average service rate is the same as in Example 3.1.3 due to the distribution of the service times. The average number of tasks $\Bar{t}$ of $t \sim 1+ Geometric(0.9)$ is $1$ plus the inverse of $p=0.9$. Therefore, the average number of tasks is 10. The average of $Uniform(0.1, 0.2)$ is $0.15$. Multiplying this and the number of tasks together results in an average service rate $\Bar{s} = 1.5$, which is the same as the service rate in Example 3.1.3.\\\\

\noindent The server utilization is the same as in Example 3.1.3 because the average interarrival times and average service rates are the same in both examples. The server utilization is a ratio of the interarrival and service time averages.

\newpage
\include{Sections/Q2/Q2_Code.tex}