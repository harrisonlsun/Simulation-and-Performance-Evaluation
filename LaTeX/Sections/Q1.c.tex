\textbf{(c) How many jobs were in the service node at t=400, and how does the computation of this number related to the proof of Theorem 1.2.1?\\\\}
There were 7 jobs in the service node at t=400. This relates to the proof of Theorem 1.2.1 in that the departure time of the job is related to the arrival time such that it is equal to the sum of the arrival time, delay time, and service time. In the case where the service node is empty at the arrival time, the delay time is 0. When the arrival event occurs, we can say that the number of jobs in the service node increments by 1. Similarly, when a departure event occurs, we can say that the number of jobs in the service node decrements by 1. Thus, the number of jobs in the node at any given time is equal to the number of jobs that arrived subtracted by the number of jobs that have left the service node up the the specified time. That is, the number of jobs in the service node is equal to the number of jobs currently being processed plus the number of jobs waiting in the queue.\\\\ 